\documentclass[a4paper,12pt]{article}
\usepackage[utf8]{inputenc}
\usepackage[T1]{fontenc}
\usepackage[hungarian]{babel}
\usepackage{graphicx}
\usepackage{geometry}
\geometry{a4paper,
		     tmargin = 30mm, 
		     lmargin = 30mm,
		     rmargin = 30mm,
		     bmargin = 30mm}
\usepackage{mathtools}
\usepackage{amsmath}
\usepackage{color}
\usepackage{setspace}
\usepackage{amsmath,amssymb}
\usepackage{float}
\usepackage{indentfirst}

\renewcommand\thesection{\Roman{section}}
\renewcommand\thesubsection{\thesection.\arabic{subsection}}

\begin{document}

\begin{titlepage}
% Template for future notes.

	\centering
	
	\includegraphics[width=0.66\textwidth]{elte.jpg}\par\vspace{1cm}
	{\scshape\LARGE ELTE TTK \par}
	{\scshape\large Fizika szak \par}
	\vspace{1cm}
	
	{\scshape\LARGE Szakdolgozat \par}
	\vspace{1cm}
	
	{\scshape\Large Koaleszcencia modell kidolgozása transzportmodellekhez\par}
	\vspace{0.2cm}
	{\Large\itshape Olar Alex\par}
	\vspace{2cm}
	
	témavezető\par
	\vspace{0.3cm}
	{\Large Wolf György}

	\vfill

	{\large 2018 \par}
\end{titlepage}

\begin{abstract}
\par Egy nehézion ütközésben résztvevő alkotó elemek száma néhány ezerig terjed legfeljebb, így a kidolgozott néhány-test elméletek, mint a három-test problémára kidolgozott Fagyejev-egyenletek, nem alkalmazhatóak, de az alkotóelemek alacsony száma miatt még a statisztikus fizikai modellek sem használhatóak, ráadásul nem is egyensúlyi reakciókról van szó az esetek többségében.
\par A rendelkezésre álló számítási kapacitás lehetővé tette, egy-egy ilyen nemegyensúlyi reakció teljes vizsgálatát, mikroszkópikus transzport-modellek segítségével. Egy ilyen modell a Boltzmann-Uehling-Uhlenbeck elmélet (BUU), ami fázistérben leírja adott részecskék között az ütközéseket és figyelembe veszi az azok között ható kölcsönhatást, egy időfüggő, átlagtér potenciállal. Korai modellek a részecskéket szabadnak tekintették, amikor azok nem vettek részt ütközésekben.
\par Az én célom, hogy egy, a BUU-ra épülő szimulációhoz kidolgozzak egy olyan programot, ami a kölcsönható részecskéket, esetemben főként nukleonokat, klaszterezi, azaz csomósodásokat keres különböző távolság definíciók mellett (térben, impulzustérben, stb.). Ennek fontos szerepe lehet a detektor válasz meghatározásakor, 
\end{abstract}

\vfill

\newpage
\tableofcontents
\newpage

\section{Elméleti áttekintés}

\subsection{Bevezető}

\par Egy nehézion ütközés erősen nem egyensúlyi termodinamikai rendszer. A statisztikus fizikában $10^{23}$ részecskére jól kidolgozott, statisztikus modell áll rendelkezésünkre, továbbá jól tudjuk magyarázni a néhányrészecske rendszereket is, azonban például egy $Au+Au$ ütközésben a részecskék száma még és már nem kezelhető a korábbi modellekkel.

\par Kezdetben a folyamat leírására termodinamikai modelleket állítottak fel, amelyekben különböző hipotéziseket tettek fel. Ezek közé tartozott, hogy a részecskék gyorsan termalizálódnak és kialakul egy globális egyensúly, és már egyensúlyi állapotukban detektáljuk őket. Eztután hidrodinamikai modellekhez folyamodtak amelyekben már nem volt globális, csak lokális termodinamikai egyensúly.

\par Azonban egy prominensebb ága a nehézion ütközések leírásának a nemegyensúlyi, mikroszkopikus transzport-modellek. Először kaszkád elméleteket dolgoztak ki, amelyben a részecskék között csak ütközéskor hatottak kölcsön, később azonban hosszú hatótávolságú erőket és nukleáris potenciálokat is figyelembe vettek.

\section{Transzport egyenletek}

\par A nehézion ütközések dinmaikáját transzport egyenletek segítségével lehet vizsgálni. Ennek két fő irányzata van, az egyik a Boltzmann-modellre épülő hidrodinamikai megközelítés, ami szerint 

\begin{equation}
	N = \int d^{3}\vec{p} \int d^{3}\vec{r} \quad f(\vec{r}, \vec{p}, t)
\end{equation}

ahol N a részecskék száma, míg $f(\vec{r}, \vec{p}, t)$ a fázistérben vett sűrűség függvény. Mivel a fázistérfogatelem ( $d^{3}\vec{p}\cdot d^{3}\vec{r}$ ) állandó, azonban ütközés során a részecske sűrűség változik, így az csak a fázissűrűségen keresztül változhat. Így tehát kapjuk a Boltzmann-egyenletet

\begin{equation}
	\frac{df}{dt} = \frac{\partial f}{\partial t} + \frac{\partial f}{\partial \vec{p}}\frac{d\vec{p}}{dt} + \frac{\partial f}{\partial \vec{r}}\frac{d\vec{r}}{dt} = I_{coll}
\end{equation}

\par Ezt pedig a szokott alakra hozva, bevezethetünk egy $I_{coll}$ ütközési integrált, amire különböző hipotéziseket tehetünk majd fel.

\begin{equation}
	\frac{\partial f}{\partial t} + \frac{\partial f}{\partial \vec{p}}\vec{F} + \vec{\nabla}f\frac{\vec{p}}{m} = I_{coll}
\end{equation}

\par Ez még természetesen csak az alapvető fizikai modell, a transzport-modellhez az úgynevezett Boltzmann-Uehling-Uhlenbeck egyenleteket használják. Tehát az előbbi egyenletbe bevezetnek egy impulzusfüggő átlagtéret $U(\vec{r}, \vec{p})$. Alacsony energiákon a rugalmatlan ütközések elhanyagolhatóak, a rendszer csak nukleonokból áll. 
	
\begin{gather}
	m^{*}(\vec{r}, \vec{p}) = m_{N} + U(\vec{r}, \vec{p}) \quad \quad E^{2} = m^{* 2} + p^{2} \\ 
	\frac{\partial H}{\partial q_{i}} = -\frac{dp_{i}}{dt} \quad \quad \frac{\partial H}{\partial p_{i}} =\frac{dq_{i}}{dt}
\end{gather}
	
\par Ahol tömeghéjon lévő kvázi-részecske közelítéssel élve az egyenlet átfogalmazható, felhasználva a Hamilton-egyenleteket:

\begin{equation} \label{BUU-eq:1}
	\frac{\partial f}{\partial t} + \frac{\partial f}{\partial \vec{p}}\Big(-\frac{m^{*}}{E}\nabla_{\vec{r}}U\Big) + \frac{\partial f}{\partial \vec{r}}\Big(\frac{\vec{p}}{E}+\frac{m^{*}}{E}\nabla_{\vec{p}}U\Big) = I_{coll}
\end{equation}

\par Az ütközési integrál ( $I_{coll}$ ) kvantumos jelenségek közül csak a Pauli-elvet veszi figyelembe, így jelentős részben klasszikus fizikán alapszik. Az így kapott egyenletet \eqref{BUU-eq:1} Boltzmann-Uehling-Uhlenbeck-egyenletnek nevezik.

\subsection{ Numerikus megoldás}

\par A témavezetőm szimulációs kódja a szokásos módszerrel áll neki ennek az integro-differenciál egyenlet megoldásának. A folytonos eloszlásfüggvény helyettesíthető véges számú pontrészecskékkel, matematikailag Dirac-delta disztribúciókkal. A fluktuációk simítása érdekében $N$ párhuzamos eseményt vizsgálva az eloszlás függvény $A$ nukleonra

\begin{equation}
	f(\vec{r}, \vec{p}, t) = \frac{1}{N}\sum_{i}^{N \times A} \delta(\vec{r} - \vec{r}_{i}(t))\delta(\vec{p} - \vec{p}_{i}(t))
\end{equation}

\par Ezt bevezetve látható, hogy a modell leegyszerűsödött pontrészecskék mozgására. A mozgásegyenletek a Hamilton-egyenletekből kaphatóak, amelyeket a \eqref{BUU-eq:1}-ben is felhasználtam.

\par A transzport-modell nem tartalmaz szabd paramétereket. A deriváltakat differenciálok váltják fel. Az egyrészecske mozgásegyenlet megoldásához a prediktor-korrektor módszer a következő

\begin{gather}
	\vec{p}^{pr}_{i} = \vec{p}_{i} - \Delta t \frac{\partial H}{\partial \vec{r}_{i}} \quad \quad \vec{r}^{pr}_{i} = \vec{r}_{i} + \Delta t \frac{\partial H}{\partial \vec{p}_{i}}
\end{gather}

\par Ahol $^{pr}$ jelöli a prediktált helyet és impulzust. Ez még korrekcióra szorul, ehhez ki kell számolni az predikdált helyen a Hamilton-függvény értékét, majd a helyet és momentumot annak megfelelően léptetni

\begin{gather}
	\vec{p}^{co}_{i} = \vec{p}_{i} - \frac{\Delta t}{2} \Big(\frac{\partial H}{\partial \vec{r}_{i}} + \frac{\partial H^{pr}}{\partial \vec{r}_{i}^{pr}} \Big) \quad \quad \vec{r}^{co}_{i} = \vec{r}_{i} + \frac{\Delta t}{2} \Big( \frac{\partial H}{\partial \vec{p}_{i}} + \frac{\partial H^{pr}}{\partial \vec{p}_{i}^{pr}} \Big)
\end{gather}

\par Az átlagtér potenciál szabadon választható az éppen megfelelő elméleti megfontolások alapján. 

\end{document}